%% -*- LaTeX -*-
%%
%% LaTeX file for mathematical extensions and support macros
%% Needs AMS-Math to work correct.
%
%% Copyright (c) Ralf Hoppe
%%
%% $Id: math-inc.tex,v 1.1 2003-08-09 11:59:33 ralf Exp $
%%
%% History:
%% $Log: not supported by cvs2svn $
%%



%% Macros and Definitions
%%

\usepackage{array}
\usepackage{ifthen}
\usepackage{algorithmic}
\usepackage{algorithm}
\usepackage{amsopn,amsthm,amsfonts}


\addtolength{\extrarowheight}{2pt}      %% extra vertical space in array environment
\addtolength{\jot}{2pt}                 %% extra vertical space in eqnarray environment


\floatname{algorithm}{Algorithmus}      %% from algorithm package


%%%%%%%%%%%%%%%%% New commands


%% \newcommand{\tabvstretch} {\renewcommand{\arraystretch}{1.5}} % table/array vertical stretching
%% \newcommand{\tabvnormal}  {\renewcommand{\arraystretch}{1.0}} % reset to default

\newcommand{\mbig}[1]{\displaystyle{#1}}
\newcommand{\mtab}[1]{                                        % math in tables
        \setlength{\fboxrule}{0pt}                            %
        \fbox{\ensuremath{\mbig{#1}}}}


%% Optional superscript if exponent is unequal to zero
%% Requires \usepackage{ifthen}
%% Example: \optsup {0}{x} displays "x", \optsup {n}{x} displays "x^n"
\newcommand{\optsup} [2] {{#2} \ifthenelse{\equal{#1}{0}}{\mbox{}}{^{#1}}}

%% Assigns a value to variable (used in algorithms)
\newcommand{\assign} [2] {\ensuremath{{#1} \Leftarrow {#2}}} % variable := value

\newcommand{\incircle} [1] {\ensuremath{\mathbin{\settowidth{\dimen6}{\mbox{$\bigcirc$}}%
              \makebox[0pt][l]{$\bigcirc$}\makebox[\dimen6]{\small{#1}}}}}

\newcommand{\msp} {\ensuremath{\;}}     %% math space


%%%%%%%%%%%%%%%%% Parameters of elliptic integrals and functions

\newcommand{\tpar} [3] {#1; #2; #3}
\newcommand{\epar} [2] {#1; #2}
\newcommand{\spar} [1] {#1}


%%%%%%%%%%%%%%%%% New functions


\DeclareMathOperator{\ipart}{\Im} %% Imag
\DeclareMathOperator{\rpart}{\Re} %% Real


%% Simple Polynomial
%% Example: \poly{a}{x}{n} defines a polynomial in x with coefficients a[i]
%% from i=0 to i=n

\newcommand {\poly} [3] {\ensuremath{{#1}_{#3}{#2}^{#3}+{#1}_{{#3}-1}{#2}^{{#3}-1}+\cdots +{#1}_2 {#2}^2+{#1}_1 {#2}+{#1}_0}}

%% Basic functions

\DeclareMathOperator{\sech}{sech} %% 1/cosh(x)
\DeclareMathOperator{\arsinh}{arsinh}
\DeclareMathOperator{\arcosh}{arcosh}
\DeclareMathOperator{\artanh}{artanh}
\DeclareMathOperator{\gd}{gd} %% Guderman function gd(x) = arctan(sinh(x))


%% Algebraic functions
\DeclareMathOperator{\GF}{GF}   %% Galois field


%% Chebyshev functions (polynomials)
\DeclareMathOperator{\T}{T}
\DeclareMathOperator{\U}{U}


%% Arithmetic Geometric Mean (AGM)
\DeclareMathOperator{\AGM}{M}

%% Basic definition for abstract Jacobian elliptic function pq
%% 1. arg: exponent
%% 2. arg: name
%% 3. arg: argument
%% 4. arg: module (k)

\newcommand{\epq} [4] { \optsup{#1}{#2} \left( \epar{#3}{#4} \right)}
\newcommand{\exarg} [1] {\underset{{}^{\scriptscriptstyle{\star}}}{#1}}


%% all jacobian elliptic functions

\DeclareMathOperator{\jam}{am}
\DeclareMathOperator{\jdelta}{\varDelta} % delta-amplitude
\DeclareMathOperator{\jsn}{sn}
\DeclareMathOperator{\jcn}{cn}
\DeclareMathOperator{\jdn}{dn}
\DeclareMathOperator{\jcd}{cd}
\DeclareMathOperator{\jdc}{dc}
\DeclareMathOperator{\jsc}{sc}
\DeclareMathOperator{\jnc}{nc}
\DeclareMathOperator{\jds}{ds}
\DeclareMathOperator{\jcs}{cs}
\DeclareMathOperator{\jsd}{sd}
\DeclareMathOperator{\jns}{ns}
\DeclareMathOperator{\jnd}{nd}

%% Example \edc [2]{u}{k}
%%         \edc {u}{k}

\newcommand{\efunc} [4] [0] {\epq{#1}{#2}{#3}{#4}} % general elliptic function

\newcommand{\esn} [3] [0] {\epq{#1}{\jsn}{#2}{#3}}
\newcommand{\ecn} [3] [0] {\epq{#1}{\jcn}{#2}{#3}}
\newcommand{\edn} [3] [0] {\epq{#1}{\jdn}{#2}{#3}}
\newcommand{\ecd} [3] [0] {\epq{#1}{\jcd}{#2}{#3}}
\newcommand{\edc} [3] [0] {\epq{#1}{\jdc}{#2}{#3}}
\newcommand{\esc} [3] [0] {\epq{#1}{\jsc}{#2}{#3}}
\newcommand{\enc} [3] [0] {\epq{#1}{\jnc}{#2}{#3}}
\newcommand{\eds} [3] [0] {\epq{#1}{\jds}{#2}{#3}}
\newcommand{\ecs} [3] [0] {\epq{#1}{\jcs}{#2}{#3}}
\newcommand{\esd} [3] [0] {\epq{#1}{\jsd}{#2}{#3}}
\newcommand{\ens} [3] [0] {\epq{#1}{\jns}{#2}{#3}}
\newcommand{\ejnd} [3] [0] {\epq{#1}{\jnd}{#2}{#3}} % \end is reserved by LaTeX

\newcommand{\eam} [3] [0] {\epq{#1}{\jam}{#2}{#3}}

\newcommand{\edelta} [3] [0] {\epq{#1}{\jdelta}{#2}{#3}}
\newcommand{\exdelta} [3] [0] {\epq{#1}{\exarg{\jdelta}}{#2}{#3}}


%% Complete Elliptic Integral

\DeclareMathOperator{\jK}{K}

\newcommand{\eK} [2] [0] { \optsup{#1}{\jK} \left(\spar{#2}\right)}


%% Incomplete Elliptic Integral

\DeclareMathOperator{\F}{F}
\DeclareMathOperator{\xF}{\exarg{\F}}
\DeclareMathOperator{\E}{E}
\DeclareMathOperator{\jPi}{\Pi}

\newcommand{\eF} [3] [0] {\epq{#1}{\F}{#2}{#3}}          %% Legendre form F(phi; k)
\newcommand{\exF} [3] [0] {\epq{#1}{\xF}{#2}{#3}}        %% F(x; k)
\newcommand{\eE} [3] [0] {\epq{#1}{\E}{#2}{#3}}          %% E(phi; k)
\newcommand{\ePi} [3] {\jPi (\tpar{#1}{#2}{#3})}



%% Zolotarev's function

\DeclareMathOperator{\Z}{Z}

\newcommand{\Zo} [3] [0] {\epq{#1}{\Z}{#2}{#3}}


%% Zeroes and Poles
\newcommand{\zn} [3] [0] {\optsup{#1}{{#2}_{\! \scriptscriptstyle 0}}_{#3}}
\newcommand{\pn} [3] [0] {\optsup{#1}{{#2}_{\! \scriptscriptstyle \infty}}_{#3}}

