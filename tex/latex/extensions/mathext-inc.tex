%% -*- LaTeX -*-
%%
%% LaTeX file for higher math. extensions (to be included from another file,
%% which must provide the required packages)
%%
%% Copyright (C) 2012-2020 Ralf Hoppe <rho@dfcgen.de>
%%
%% Requires:
%% \makeatletter               % at beginning of preamble
%% \usepackage{amsopn,amssymb} % \DeclareMathOperator, AMS symbols
%% \usepackage{ifthen}         % \ifthenelse
%% \makeatother                % at end of preamble
%%


%%%%%%%% HTML/PDF Conditionals

%% Fractions look extremely ugly in tables, because there is no space over
%% and under. The following macro writes a fbox without a frame but with the
%% fboxsep. But this might not work on some LaTeX to HTML converters.
\@ifpackageloaded{tex4ht}{%
  \newcommand{\mtab} [1] { %% math in tables
    \ensuremath{\mbig{#1}}}}{%
  \newcommand{\mtab} [1] {%
    \setlength{\fboxrule}{0pt}%
    \fbox{\ensuremath{\mbig{#1}}}}}


%% Inline math in headings/titles
%% Usage: \tocmath{expr}{replacement}
%% Example: \tocmath{x^y}{x times y}
\AtBeginDocument{
  \@ifpackageloaded{hyperref}{ %% hyperref loaded (PDF)
    \newcommand{\tocmath} [2] {%
      \texorpdfstring{%
        \mathversion{bold}%
        \ensuremath{#1}%
        \mathversion{normal}
      }{#2}}%
  }{ %% hyperref NOT loaded (no replacement)
    \newcommand{\tocmath} [2] {\ensuremath{#1}}}
}



%%%%%%%% Misc

\newcommand{\msp} {\ensuremath{\;}}% math space
\newcommand{\mperiod} {\msp.}      % period (at end of sentence) in math mode

\newcommand{\mbig} [1]   {\displaystyle{#1}} % FIXME: try to remove macro \mbig
\newcommand{\mboxed} [1] {\boxed{#1}}
\newcommand{\mvec} [1] {\ensuremath{\mathbf{#1}}} % Bold matrix/vector
\newcommand{\mdet} [1] {\mvec{#1}} % determinant variable

%% Optional superscript if exponent is unequal to zero
%% Example: \optsup {0}{x} displays "x", \optsup {n}{x} displays "x^n"
\newcommand{\optsup} [2] {{#2} \ifthenelse{\equal{#1}{0}}{\mbox{}}{^{#1}}}


\newcommand{\incircle} [1] {\ensuremath{\mathbin{\settowidth{\dimen6}{\mbox{$\bigcirc$}}%
              \makebox[0pt][l]{$\bigcirc$}\makebox[\dimen6]{\small{#1}}}}}


\AtBeginDocument{
  \ifx\opentimes\undefined % \opentimes from px/txfonts packages loaded?
  \def\opentimes{\times}
  \fi
}



%%%%%%%% Analysis

\newcommand{\pdc} {\partial}            % partial differential character
\newcommand{\tdc} {\mathrm{d}}          % total differential character
\newcommand{\pic} {\,\pdc}              % partial differential with leading (thin) space
\newcommand{\tic} {\,\tdc}              % total differential with leading (thin) space

\newcommand{\pdiff} [3] [0] {\frac{\optsup{#1}{\pdc} #2}{\pdc \optsup{#1}{#3}}}
\newcommand{\tdiff} [3] [0] {\frac{\optsup{#1}{\tdc} #2}{\tdc \optsup{#1}{#3}}}

\newcommand{\vp} {\,\textup{V.\,P.}}    % Cauchy principal value
\DeclareMathOperator{\residue}{res}     % residue
\newcommand{\res} [2] {\residue_{#1} {#2}} % residue of function #2 at point #1

\newcommand{\zn} [3] [0] {\optsup{#1}{{#2}\kern-.24em\lower.68ex\hbox{$\scriptscriptstyle\circ$}}_{#3}}
\newcommand{\pn} [3] [0] {\optsup{#1}{{#2}\kern-.24em\lower.68ex\hbox{$\scriptscriptstyle\opentimes$}}_{#3}}

% polynomial: \poly{a}{x}{n} defines a polynomial in x with coefficients a[i]
% from i=0 to i=n
\newcommand {\poly} [3] {\ensuremath{{#1}_{#3}{#2}^{#3}+\cdots +{#1}_2 {#2}^2+{#1}_1 {#2}+{#1}_0}}



%%%%%%%% Complex Numbers

\newcommand{\ji} {\ensuremath{\mathrm{j}}}

%% Conjugate complex value of an expression
%%
%% Indicates an expression, or it's (optional) "prefix representative",
%% as a conjugate complex value.
%%
\newcommand{\cjgt} [2] [empty] {%
  \def\tempa{empty}%
  \def\tempb{#1}%
  \ifx\tempb\tempa
    {#2}^\ast%
  \else
    {#1}^\ast{#2}%
  \fi
}

\DeclareMathOperator{\imop}{Im}         % imaginary part of complex number (Im{} operator)
\DeclareMathOperator{\reop}{Re}         % real part of complex number (Re{} operator)
\DeclareMathOperator{\ang}{\measuredangle} % angle of a complex number (alt. \angle, \arg)



%%%%%%%% Functions

\DeclareMathOperator{\expf}{e}          % Exponential function
\DeclareMathOperator{\sign}{sgn}        % Sign of an expression
\DeclareMathOperator{\cosec}{cosec}     % 1/sin(x)
\DeclareMathOperator{\sech}{sech}       % 1/cosh(x)
\DeclareMathOperator{\arsinh}{arsinh}
\DeclareMathOperator{\arcosh}{arcosh}
\DeclareMathOperator{\artanh}{artanh}
\DeclareMathOperator{\ld}{\log_{2}}     % logarithmus dualis (ld)
\DeclareMathOperator{\gd}{gd}           % Guderman function gd(x) = arctan(sinh(x))
\DeclareMathOperator{\si}{si}           % si(x) = sin(x)/x
\DeclareMathOperator{\sinc}{sinc}       % sinc(x) = sin(\pi x)/(\pi x)
\DeclareMathOperator{\tn}{T}            % Chebyshev polynomial of 1st kind
\DeclareMathOperator{\un}{U}            % Chebyshev polynomial of 2nd kind


%% Probability
\newcommand{\avg} [1] {\overline{#1}}   % Average (in German: Mittelwert)
\DeclareMathOperator{\pr}{\Pr}          % Probability (sometimes only P)
\DeclareMathOperator{\cnorm}{\Phi}      % Normal (cummulative) distribution
\DeclareMathOperator{\mean}{E}          % Expected value (operator)
\DeclareMathOperator{\var}{Var}         % Variance, e.g. \var X
\DeclareMathOperator{\cov}{Cov}         % Covariance
\DeclareMathOperator{\corr}{Corr}       % Correlation coefficient


%% Algebra
\DeclareMathOperator{\lcm}{lcm}         % Least Common Multiplier
\DeclareMathOperator{\ord}{ord}         % Order
\DeclareMathOperator{\GF}{GF}           % Galois field
\DeclareMathOperator{\trace}{tr}        % trace
\DeclareMathOperator{\htrace}{htr}      % half-trace
\DeclareMathOperator{\MONTG}{M}         % Montgomery transformation
\DeclareMathOperator{\totient}{\phi}    % Euler's totient function
\newcommand{\rcls} [2] {{\left[ {#1} \right] }_{#2}} % residue class
\newcommand{\mequiv} {\ensuremath{\stackrel{\scriptscriptstyle\wedge}{=}}} % equivalent: '=' with hat
