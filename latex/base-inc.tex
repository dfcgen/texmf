%% -*- LaTeX -*-
%%
%% LaTeX include file with base macros and packages
%%
%% Copyright (C) 2007-2011 Ralf Hoppe
%%
%% $Id$
%%

\usepackage{ifthen}
\usepackage{newclude}
\usepackage{bibgerm}
\usepackage{textcomp}
\usepackage{pifont}
%% \usepackage{path} %% package 'path' is obsolete (use url instead)


%% definition of \provideenvironment{name}[narg]{begdef}{enddef} command
\def\provide@environment#1{%
  \@ifundefined{#1}{\relax}{%
  \expandafter\let\csname#1\endcsname\relax
  \expandafter\let\csname end#1\endcsname\relax
  \new@environment{#1}}}

%% You can derive some  \ifisdefined{} ... \else...\fi construct from this
\def\isdefined#1{\begingroup
    \expandafter\endgroup\expandafter\ifx
    \csname #1\endcsname\@undefined
    undefined\else defined\fi}


%%
%% Checks if a TeX job is equal to (file) name. This is tricky because
%% TeX macro \jobname expands to catcode 12 (other) characters.
%%
%% Example usage:
%%   \if\thisjob{name}%
%%   do this%
%%   \else%
%%   do that%
%%   \fi%
%% 
\def\thisjob#1{..\fi\begingroup
  \edef\jobA{\jobname}%
% Convert an arbitrary string to "other" characters:
% do \csname, \string, then \@gobble
  \edef\jobB{\expandafter \expandafter \expandafter \@gobble
   \expandafter \string \csname #1\endcsname}%
  \expandafter \endgroup \ifx\jobA\jobB}


%% Index
\renewcommand{\indexname}{Stichwortverzeichnis}



%% Uncomment the following lines to suppress additional text added to a cite
%% command as optional argument (normally specific part/chapter/section number
%% links into the referenced work/paper
%%
%% \let\origcite\cite
%% \renewcommand{\cite} [2] [ignored] {\protect\origcite{#2}}



%%%%%%%%%%%%%%%%%%%%%%%%%%%%%%%%%%%%%%%%%%%%%%%%%%
%% Compatibility with older versions
%%%%%%%%%%%%%%%%%%%%%%%%%%%%%%%%%%%%%%%%%%%%%%%%%%
%% \providecommand{\foreignlanguage} [2] {#2} %% babel
%% \providecommand{\printnomenclature}{\printglossary} %% nomencl/TeX4ht
%% \providecommand{\makenomenclature}{\makeglossary}
%% \providecommand{\thenomenclature}{\theglossary}
%% \providecommand{\theglossary}{\thenomenclature}
