%% -*- LaTeX -*-
%%
%% LaTeX file for elliptic functions, to be included from another file (which
%% provides the required packages)
%%
%% Copyright (C) 2012 Ralf Hoppe <ralf.hoppe@ieee.org>
%%
%% $Id$
%%
%% Requires:
%% \usepackage{amsopn}   % \DeclareMathOperator
%% \usepackage{ifthen}   % \ifthenelse
%% \input{mathext}       % \optsup, \ji
%%


%% Elliptic Parameters
\newcommand{\tpar} [3] {#1;\, #2; #3}
\newcommand{\epar} [2] {#1;\, #2}
\newcommand{\spar} [1] {#1}
\newcommand{\exarg} [1] {\underset{{}^{\scriptscriptstyle{\star}}}{#1}}


\DeclareMathOperator{\AGM}{M} % Arithmetic Geometric Mean


%% Complete Elliptic Integral
\DeclareMathOperator{\jK}{K}  % Complete Elliptic Integral
\newcommand{\eK} [2] [0] { \optsup{#1}{\jK} \left(\spar{#2}\right)}


%% Preliminaries for Jacobian elliptic functions
%% Note: \the\epqheight prints the height value
\newlength{\epqheight}
\newlength{\epqlimit}
\settoheight{\epqlimit}{\ensuremath{\epar{\ji\xi/2+\jK'}{k'}}}
\newsavebox{\epqbox}

%% Basic definition for abstract Jacobian elliptic function pq
%% 1. arg: exponent
%% 2. arg: name
%% 3. arg: argument
%% 4. arg: module (k)
\newcommand{\epq} [4] {%
  \savebox{\epqbox}{\ensuremath{\epar{#3}{#4}}}
  \settoheight{\epqheight}{\usebox{\epqbox}}
  \optsup{#1}{#2}%
  \ifthenelse{\lengthtest{\epqheight > \epqlimit}}%
             {\left(\usebox{\epqbox}\right)}%
             {(\usebox{\epqbox})}%
}


%% Incomplete Elliptic Integral
\DeclareMathOperator{\ellF}{F}
\DeclareMathOperator{\ellxF}{\exarg{\ellF}}
\DeclareMathOperator{\ellE}{E}
\DeclareMathOperator{\ellPi}{\Pi}

\newcommand{\eF} [3] [0] {\epq{#1}{\ellF}{#2}{#3}}       %% Legendre form F(phi; k)
\newcommand{\exF} [3] [0] {\epq{#1}{\ellxF}{#2}{#3}}     %% F(x; k)
\newcommand{\eE} [3] [0] {\epq{#1}{\ellE}{#2}{#3}}       %% E(phi; k)
\newcommand{\ePi} [3] {\ellPi (\tpar{#1}{#2}{#3})}


%% all Jacobian elliptic functions
\DeclareMathOperator{\jam}{am}
\DeclareMathOperator{\jdelta}{\varDelta} % delta-amplitude
\DeclareMathOperator{\jsn}{sn}
\DeclareMathOperator{\jcn}{cn}
\DeclareMathOperator{\jdn}{dn}
\DeclareMathOperator{\jcd}{cd}
\DeclareMathOperator{\jdc}{dc}
\DeclareMathOperator{\jsc}{sc}
\DeclareMathOperator{\jnc}{nc}
\DeclareMathOperator{\jds}{ds}
\DeclareMathOperator{\jcs}{cs}
\DeclareMathOperator{\jsd}{sd}
\DeclareMathOperator{\jns}{ns}
\DeclareMathOperator{\jnd}{nd}

%% Example \edc [2]{u}{k}
%%         \edc {u}{k}
\newcommand{\efunc} [4] [0] {\epq{#1}{#2}{#3}{#4}} % general elliptic function

\newcommand{\esn} [3] [0] {\epq{#1}{\jsn}{#2}{#3}}
\newcommand{\ecn} [3] [0] {\epq{#1}{\jcn}{#2}{#3}}
\newcommand{\edn} [3] [0] {\epq{#1}{\jdn}{#2}{#3}}
\newcommand{\ecd} [3] [0] {\epq{#1}{\jcd}{#2}{#3}}
\newcommand{\edc} [3] [0] {\epq{#1}{\jdc}{#2}{#3}}
\newcommand{\esc} [3] [0] {\epq{#1}{\jsc}{#2}{#3}}
\newcommand{\enc} [3] [0] {\epq{#1}{\jnc}{#2}{#3}}
\newcommand{\eds} [3] [0] {\epq{#1}{\jds}{#2}{#3}}
\newcommand{\ecs} [3] [0] {\epq{#1}{\jcs}{#2}{#3}}
\newcommand{\esd} [3] [0] {\epq{#1}{\jsd}{#2}{#3}}
\newcommand{\ens} [3] [0] {\epq{#1}{\jns}{#2}{#3}}
\newcommand{\ejnd} [3] [0] {\epq{#1}{\jnd}{#2}{#3}} % \end is reserved by LaTeX

\newcommand{\eam} [3] [0] {\epq{#1}{\jam}{#2}{#3}}
\newcommand{\edelta} [3] [0] {\epq{#1}{\jdelta}{#2}{#3}}


%% Zolotarev's function
\DeclareMathOperator{\Z}{Z}
\newcommand{\Zo} [3] [0] {\epq{#1}{\Z}{#2}{#3}}

