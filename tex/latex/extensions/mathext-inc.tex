%% -*- LaTeX -*-
%%
%% LaTeX file for mathematical symbols, to be included from another file (which
%% provides the required packages)
%%
%% Copyright (C) 2012, 2020 Ralf Hoppe <ralf.hoppe@ieee.org>
%%
%% $Id$
%%
%% Requires:
%% \usepackage{amsopn,amssymb} %% DeclareMathOperator, AMS symbols
%% \usepackage{trfsigns}       %% symbols for FT, LT and e
%% \usepackage{ifthen}         %% \ifthenelse
%%



%% Misc


%% Optional superscript if exponent is unequal to zero
%% Example: \optsup {0}{x} displays "x", \optsup {n}{x} displays "x^n"
\newcommand{\optsup} [2] {{#2} \ifthenelse{\equal{#1}{0}}{\mbox{}}{^{#1}}}


\newcommand{\incircle} [1] {\ensuremath{\mathbin{\settowidth{\dimen6}{\mbox{$\bigcirc$}}%
              \makebox[0pt][l]{$\bigcirc$}\makebox[\dimen6]{\small{#1}}}}}


\AtBeginDocument{
  \ifx\opentimes\undefined %% \opentimes from px/txfonts packages loaded?
  \def\opentimes{\times}
  \fi
}



%% Analysis


%% Conjugate complex value of an expression
%%
%% Indicates an expression, or it's (optional) "prefix representative",
%% as a complex value.
%%
\newcommand{\cjgt} [2] [empty] {%
  \def\tempa{empty}%
  \def\tempb{#1}%
  \ifx\tempb\tempa
    {#2}^\ast%
  \else
    {#1}^\ast{#2}%
  \fi
}


\newcommand{\res} [2] {\residue_{#1} {#2}}  %% residue of function #2 at point #1

\newcommand{\zn} [3] [0] {\optsup{#1}{{#2}\kern-.24em\lower.68ex\hbox{$\scriptscriptstyle\circ$}}_{#3}}
%% \newcommand{\zn} [3] [0] {\optsup{#1}{{#2}_{\! \scriptscriptstyle \circ}}_{#3}}
\newcommand{\pn} [3] [0] {\optsup{#1}{{#2}\kern-.24em\lower.68ex\hbox{$\scriptscriptstyle\opentimes$}}_{#3}}
%% \newcommand{\pn} [3] [0] {\optsup{#1}{{#2}_{\! \scriptscriptstyle \infty}}_{#3}}


\newcommand{\ji} {\im} %% j from package trfsigns
\newcommand{\lps} {s}  %% Laplace variable s or p


%% \DeclareMathOperator{\imop}{\Im}
\DeclareMathOperator{\imop}{Im}         %% Imaginary part of complex number (Im{} operator)
%% \DeclareMathOperator{\reop}{\Re}
\DeclareMathOperator{\reop}{Re}         %% real part of complex number (Re{} operator)
\DeclareMathOperator{\ang}{\measuredangle}    %% angle of a complex number
%% \DeclareMathOperator{\ang}{\angle}   %% angle of a complex number
%% \DeclareMathOperator{\ang}{\arg}     %% angle of a complex number
\DeclareMathOperator{\residue}{res}     %% residue
\DeclareMathOperator{\expf}{e}          %% Exponential function



%% Basic functions

\DeclareMathOperator{\sign}{sgn}        %% Sign of an expression
\DeclareMathOperator{\cosec}{cosec}     %% 1/sin(x)
\DeclareMathOperator{\sech}{sech}       %% 1/cosh(x)
\DeclareMathOperator{\arsinh}{arsinh}
\DeclareMathOperator{\arcosh}{arcosh}
\DeclareMathOperator{\artanh}{artanh}
\DeclareMathOperator{\ld}{\log_{2}}     %% logaritmus dualis (ld)
\DeclareMathOperator{\gd}{gd}           %% Guderman function gd(x) = arctan(sinh(x))

%% Probability
\newcommand{\avg} [1] {\overline{#1}}   %% Average (in German >>Mittelwert<<)
\DeclareMathOperator{\pr}{\Pr}          %% Probability (sometimes only P)
\DeclareMathOperator{\cnorm}{\Phi}      %% Normal (cummulative) distribution
\DeclareMathOperator{\mean}{E}          %% Expected value (operator)
\DeclareMathOperator{\var}{Var}         %% Variance, e.g. \var X
\DeclareMathOperator{\cov}{Cov}         %% Covariance
\DeclareMathOperator{\corr}{Corr}       %% Correlation coefficient


%% Algebraic functions
\DeclareMathOperator{\lcm}{lcm}         %% Least Common Multiplier
\DeclareMathOperator{\ord}{ord}         %% Order
\DeclareMathOperator{\GF}{GF}           %% Galois field
\DeclareMathOperator{\trace}{tr}        %% trace
\DeclareMathOperator{\htrace}{htr}      %% half-trace
\DeclareMathOperator{\MONTG}{M}         %% Montgomery transformation
\DeclareMathOperator{\totient}{\phi}    %% Euler's totient function
\newcommand{\rcls} [2] {{\left[ {#1} \right] }_{#2}} % residue class

%% Logic
\DeclareMathOperator{\rol}{rol}         %% Rotate Left
\DeclareMathOperator{\ror}{ror}         %% Rotate Right
\DeclareMathOperator{\shl}{shl}         %% Shift Left
\DeclareMathOperator{\shr}{shr}         %% Shift Right


%% Chebyshev functions (polynomials)
\DeclareMathOperator{\Tn}{T}
\DeclareMathOperator{\Un}{U}
\DeclareMathOperator{\si}{si}           %% si(x) = sin(x)/x
\DeclareMathOperator{\sinc}{sinc}       %% sinc(x) = sin(\pi x)/(\pi x)



%% Integral transformations (FT, LT, ZT, DFT, ...) and \e, \im from trfsigns
\DeclareMathOperator{\ft} {\mathcal F}  %% Fourier-Transform
\DeclareMathOperator{\lt} {\mathcal L}  %% Laplace-Transform
\DeclareMathOperator{\zt} {\mathcal Z}  %% Z-Transform

\newcommand{\lft} [1] [] {\allowbreak\;\laplace{#1}\allowbreak\;}    %% Continuous Fourier transform (left time domain)
\newcommand{\rft} [1] [] {\allowbreak\;\Laplace{#1}\allowbreak\;}    %% Continuous Fourier transform (right time domain)
\newcommand{\lpt} [1] [] {\allowbreak\;\laplace{#1}\allowbreak\;}    %% Continuous Laplace transform (left time domain)
\newcommand{\rlt} [1] [] {\allowbreak\;\Laplace{#1}\allowbreak\;}    %% Continuous Laplace transform (right time domain)

