%% -*- LaTeX -*-
%%
%% LaTeX file for electrical science (and signal processing) symbols,
%% to be included from another file (which provides the required packages)
%%
%% Copyright (C) 2012, 2020 Ralf Hoppe <rho@dfcgen.de>
%%
%% Requires:
%% \usepackage{amsopn}          %% \DeclareMathOperator
%% \usepackage{trfsigns}        %% symbols for FT, LT and e
%%

%%%%%%%% Integral transformations (FT, LT, ZT)
\newcommand{\lps} {s}  %% Laplace variable s or p

\DeclareMathOperator{\ft} {\mathcal F}  %% Fourier-Transform
\DeclareMathOperator{\lt} {\mathcal L}  %% Laplace-Transform
\DeclareMathOperator{\zt} {\mathcal Z}  %% Z-Transform

\newcommand{\lft} [1] [] {\allowbreak\;\stackrel{\ft}{\laplace{#1}}\allowbreak\;}    %% Continuous Fourier transform (left time domain)
\newcommand{\rft} [1] [] {\allowbreak\;\stackrel{\ft}{\Laplace{#1}}\allowbreak\;}    %% Continuous Fourier transform (right time domain)
\newcommand{\lpt} [1] [] {\allowbreak\;\stackrel{\lt}{\laplace{#1}}\allowbreak\;}    %% Continuous Laplace transform (left time domain)
\newcommand{\rlt} [1] [] {\allowbreak\;\stackrel{\lt}{\Laplace{#1}}\allowbreak\;}    %% Continuous Laplace transform (right time domain)


%%%%%%%% Transfer Functions
\DeclareMathOperator{\acf}   {\varphi} % Auto (Cross) -correlation function
\DeclareMathOperator{\spd}   {S} % Spectral Power Density
\DeclareMathOperator{\drf}   {D} % Characteristic function (Drosselung)
\DeclareMathOperator{\trf}   {H} % Transfer function
\DeclareMathOperator{\atf}   {A} % Attenuation
\DeclareMathOperator{\phf}   {B} % Phase (neg. angle)
\DeclareMathOperator{\irf}   {h} % Impulse response function
\DeclareMathOperator{\usf}   {u} % Unit step function

%%%%%%%% Cyrillic Shah
\DeclareFontFamily{U}{wncy}{}
\DeclareFontShape{U}{wncy}{m}{n}{<->wncyr10}{}
\DeclareSymbolFont{mcy}{U}{wncy}{m}{n}
\DeclareMathSymbol{\Sha}{\mathord}{mcy}{"58}

%%%%%%%% Signals
\DeclareMathOperator{\rect}  {\Pi} % Rectangular pulse
\DeclareMathOperator{\dirac} {\delta} % Dirac pulse
\DeclareMathOperator{\dcomb} {\Sha} %% Dirac (sampling) comb: \sum_{k=0}^{k=\inf} \dirac(t-k T_0)
