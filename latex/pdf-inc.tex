%% -*- LaTeX -*-
%%
%% LaTeX include file for pdf support.
%%
%% PDF Setup (see file "preferences" in \LyX Dir):
%% dvips -Z -G0 -Ppdf
%% ps2pdf -dMaxSubsetPct=100 -dCompatibilityLevel=1.3 -dSubsetFonts=true -dEmbedAllFonts=true
%%
%%
%%
%% Forces Palatino font usage (in math too). Change this to Times if you
%% want (using packages mathptmx and txfonts).
%%
%%
%% Copyright (c) Ralf Hoppe
%%
%% $Id: pdf-inc.tex,v 1.7 2008-04-20 14:27:15 ralf Exp $
%%
%% History:
%%
%% $Log: not supported by cvs2svn $
%% Revision 1.6  2007-09-30 16:51:48  ralf
%% Macro mhead will be redefined or defined (whether it was defined or not)
%%
%% Revision 1.5  2007-09-21 17:54:39  ralf
%% Option clash for mathptmx avoided
%%
%% Revision 1.4  2007-09-08 16:43:24  ralf
%% Some changes wrt. tex4ht
%%
%% Revision 1.3  2006/06/18 10:53:02  ralf
%% New macro pdfsubj
%%
%% Revision 1.2  2003/10/26 20:31:22  ralf
%% New macros
%%
%% Revision 1.1  2003/08/09 11:59:33  ralf
%% Re-arranged files and contents
%%
%%

\@ifpackageloaded{tex4ht.sty}{}{ % do not confuse TeX4ht

  %% Postscript font Times in text and math formulas (based on freely
  %% distributable fonts). Disadvantage: No support of bold in math
  %% formulas (normally needed in headings, see macro \mhead in file
  %% math-inc.tex)
  %%
  \usepackage{txfonts}
  \usepackage{mathptmx}


  %% Postscript font Palatino in text and math formulas (based on freely
  %% distributable fonts).
  %%
  %% \usepackage{pxfonts}                %% Declares \opentimes symbol)
  %% \usepackage[slantedGreek]{mathpazo} %% Palatino math


  %% Math fonts from mathdesign package for usage with standard LaTeX
  %% fonts
  %%
  %% \usepackage[garamond]{mathdesign}

  %% \docsubj{title}{subject}{keywords}
  \providecommand{\docsubj} [3] {}
  \renewcommand{\docsubj} [3] {
    \hypersetup{
      pdfauthor = {\textcopyright ~ Ralf Hoppe},
      pdftitle = {#1},
      pdfsubject = {#2},
      pdfkeywords = {#3}
    }
  }
}


\usepackage[usenames]{color}

\usepackage[                    %%
    ps2pdf,                     %% another option is pdflatex
    pdfpagemode=UseOutlines,    %% generate bookmarks
    pdfview=FitH,               %%
    pdfstartview=FitH,          %%
    pdfpagelayout=SinglePage,   %%
    bookmarks=true,             %%
    bookmarksopen=true,         %%
    bookmarksnumbered=true,     %%
    bookmarksopenlevel=1,       %%
    colorlinks=true,            %% forces no border/frame around links
    breaklinks=true,            %%
    hyperindex=true,            %%
    menucolor=BrickRed,         %% color for Acrobat menu items
    pagecolor=BrickRed,         %% color for links to other pages
    anchorcolor=Black,          %% color for anchor text
    citecolor=MidnightBlue,     %% color for bibligraphical citations in text
    linkcolor=Blue,             %% color for normal internal links
    filecolor=Purple,           %% color for URLs which open local files
    urlcolor=Purple             %% color for linked URLs
] {hyperref}


%% If hyperref is loaded (after newfloat definition e.g. for algorithms,
%% see LyX help file EmbeddedObjects.lyx for details) the following
%% redefinitions are needed.
\ifx\thealgorithm\undefined %% algorithm float defined?
\else
  \newfloat{Xalgorithm}{htbp}{loa}
  \floatname{Xalgorithm}{Algorithmus}
  \newcommand{\theHalgorithm}{\theHXalgorithm}
  \renewenvironment{algorithm}[1][htbp]{\begin{Xalgorithm}[#1]}{\end{Xalgorithm}}
\fi


%% Inline math in headings/titles (define/redefine it)
%% Usage: \mhead{expr}{replacement}
%% Example: \mhead{x^y}{x times y}
\providecommand{\mhead} [2] {\ensuremath{#1}}
\renewcommand{\mhead} [2] {%
  \texorpdfstring{%
    \mathversion{bold}\ensuremath{#1}\mathversion{normal}%
  }{%
    #2%
  }%
}
